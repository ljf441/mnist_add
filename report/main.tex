%--------------------
% Packages
% -------------------
\documentclass[11pt,a4paper]{article}
\usepackage[utf8x]{inputenc}
\usepackage[T1]{fontenc}
%\usepackage{gentium}
\usepackage{mathptmx} % Use Times Font


\usepackage[pdftex]{graphicx} % Required for including pictures
\usepackage[pdftex,linkcolor=black,pdfborder={0 0 0}]{hyperref} % Format links for pdf
\usepackage{calc} % To reset the counter in the document after title page
\usepackage[numbers]{natbib}


\frenchspacing % No double spacing between sentences
\linespread{1.2} % Set linespace
\usepackage[a4paper, lmargin=0.1666\paperwidth, rmargin=0.1666\paperwidth, tmargin=0.1111\paperheight, bmargin=0.1111\paperheight]{geometry} %margins
%\usepackage{parskip}

\usepackage[all]{nowidow} % Tries to remove widows
\usepackage[protrusion=true,expansion=true]{microtype} % Improves typography, load after fontpackage is selected


%-----------------------
% Set pdf information and add title, fill in the fields
%-----------------------
\hypersetup{ 	
pdfsubject = {Machine Learning Coursework},
pdftitle = {Machine Learning Coursework},
pdfauthor = {Laura Just Fung (lj441)}
}

%-----------------------
% Begin document
%-----------------------
\begin{document} 

\section{Introduction}

MNIST is a dataset consisting of handwritten digits and their labels commonly used for training machine learning models and image processing systems \citep{deng2012mnist}. The classical problem represented by the MNIST dataset is for machine learning algorithms to be able to learn to recognise handwritten digits, no matter how they are shaped.

For this project, the problem at hand is not just for the inference pipeline to learn to recognise MNIST digits accurately, it must also learn addition.

Previous work has been done on this by Hoshen and Peleg in 2015 \citep{hoshen2015visuallearningarithmeticoperations} and Bloice et al. in 2019 \citep{bloice2019performingarithmeticusingneural}. Hoshen and Peleg focused on end-to-end visual learning of arithmetic operations using pictures, first learning on recognising the image, then learning the arithmetic operation \citep{hoshen2015visuallearningarithmeticoperations}. Bloice et al. were more concerned with .

% This is a problem first tackled by Bloice et al. in 2019, where a convolutional neural network was trained in addition using images of MNIST digit pairs \citep{bloice2019performingarithmeticusingneural}. 

The way in which the inference pipeline will do this is simple: first, a stratified dataset of paired MNIST digits will be generated, along with the label denoting their sum. The object of this dataset is to get an even distribution of digit pair permutations, such that the resulting model will not be biased to learning on the extremes. Then, a neural network will be trained and perform supervised learning on this dataset. In order to get an accurately-performing neural network, hyper-parameter tuning will also be performed using \texttt{Optuna}.

To explore this problem further, 
% docker run --gpus all -it -p 8888:8888 mnist_add-img

\section{Dataset generation}

\section{Neural network pipeline}

\section{Other inference algorithms}

\section{Weak linear classifiers}

\section{t-SNE distributions in neural networks}


\bibliographystyle{vancouver}
\bibliography{bibliography}
\end{document}
